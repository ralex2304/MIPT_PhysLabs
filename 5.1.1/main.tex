\documentclass[a4paper, 12pt]{article}
\usepackage[a4paper,top=1.5cm, bottom=1.5cm, left=1cm, right=1cm]{geometry}
\usepackage{cmap}					% поиск в PDF
\usepackage{mathtext} 				% русские буквы в формулах
\usepackage[T2A]{fontenc}			% кодировка
\usepackage[utf8]{inputenc}			% кодировка исходного текста
\usepackage[english,russian]{babel}	% локализация и переносы

\usepackage{amsmath,amssymb}
\usepackage{indentfirst}
\usepackage{longtable}
\usepackage{graphicx}
\usepackage{array}
\usepackage{float}

\usepackage{floatflt}
\usepackage{wrapfig}
\usepackage{siunitx} % Required for alignment
\usepackage{subfig}
\usepackage{multirow}
\usepackage{rotating}
\usepackage{caption}

\graphicspath{{.}}


\title{\begin{center}Лабораторная работа №5.1.1\end{center}
Экспериментальная проверка уравнения Эйнштейна для фотоэффекта и определение постоянной Планка}
\author{Рожков А. В.}
\date{\today}

\begin{document}
    \pagenumbering{gobble}
    \maketitle
    \newpage
    \pagenumbering{arabic}
    \renewcommand*{\thesubsection}{\thesection.\Alph{subsection}}

    \textbf{Цель работы:} исследовать зависимость фототока от величины задерживающего потенциала и частоты падающего излучения, что позволяет вычислить величину постоянной Планка.

    \textbf{В работе используются:} источник света, конденсор, монохроматор УМ-2, фотоэлемент с усилителем постоянного тока.

    \section{Теоретические сведения}

        Фотоэффект -- явление испускания электронов фотокатодом, облучаемым светом. Взаимодействие монохроматического света с веществом можно описывать как взаимодействие с веществом частиц, называемых фотонами, которые обладают энергией $\hbar \omega$ и импульсом $\hbar \omega /c$. При столкновении фотона с электроном фотокатода энергия фотона полностью передается электрону, и фотон прекращает свое существование. Энергетический баланс этого взаимодействия для вылетающих электронов описывается уравнением

        \begin{equation}
            \hbar \omega = E_{max} + W
            \label{energy balance}
        \end{equation}

        \begin{figure}[!ht]
            \centering
            \includegraphics[width = 0.35\textwidth]{img/I_V_th.png}
            \caption{Зависимость фототока от напряжения на аноде фотоэлемента}
            \label{pict I(V)}
        \end{figure}

        Здесь $E_{max}$ -- максимальная кинетическая энергия электрона после выхода из фотокатода, $W$ -- работа выхода электрона из катода. Реально энергетический спектр вылетевших из фотокатода электронов непрерывен -- он простирается от нуля до $E_{max}$.

        Для измерения энергии вылетевших фотоэлектронов вблизи фотокатода обычно располагается второй электрод (анод), на который подается задерживающий ($V < 0$) или ускоряющий ($V > 0$) потенциал. При достаточно больших ускоряющих напряжениях фототок достигает насыщения (рис. \ref{pict I(V)}): все испущенные электроны попадают на анод.

        При задерживающих потенциалах на анод попадают лишь электроны, обладающие достаточно большой кинетической энергией, в то время как медленно движущиеся электроны заворачиваются полем и возвращаются на катод. При некотором значении $V = -V_0$ (потенциал запирания) даже наиболее быстрые фотоэлектроны не могут достичь анода.

        Максимальная кинетическая энергия $ E_{max} $ электронов связана с запирающим потенциалом $V_0$ очевидным соотношением $E_{max} = eV_0$. Тогда \eqref{energy balance} примет вид, называемый уравнением Эйнштейна:

        \begin{equation}
            eV_0 = \hbar\omega - W
            \label{Einstein}
        \end{equation}

        Чтобы определить величину запирающего напряжения, нам надо правильно экстраполировать получаемую токовую зависимость к нулю, т. е. определить, какова функциональная зависимость $I(V)$. Расчет для простейшей геометрии -- плоский катод, освещаемый светом, и параллельный ему анод -- приводит к зависимости

        \begin{equation}
            \sqrt{I} \propto V_0 - V,
            \label{sqrt I = V}
        \end{equation}
        т. е. корень квадратный из фототока линейно зависит от запирающего напряжения. Эта зависимость хорошо описывает экспериментальные данные.

        В работе изучается зависимость фототока из фотоэлемента от величины задерживающего потенциала $V$ для различных частот света $\omega$, лежащих в видимой области спектра.

        \begin{figure}[!ht]
            \centering
            \includegraphics[width = 0.3\textwidth]{img/V_omega_th.png}
            \caption{Зависимость запирающего потенциала от частоты света}
            \label{pict V(w)}
        \end{figure}

        Потенциал запирания $V_0$ для любого катода линейно зависит от частоты света $\omega$. По наклону прямой на графике $V_0(\omega)$ (рис. \ref{pict V(w)}) можно определить постоянную Планка:

        \begin{equation}
            \dfrac{dV_0}{d\omega} = \dfrac{\hbar}{e}
            \label{dVdw}
        \end{equation}

        Как показывает формула \eqref{dVdw}, угол наклона прямой $V_0(\omega) $ не зависит от рода вещества, из которого изготовлен фотокатод. От рода вещества, однако, зависит величина фототока, работа выхода $W$ и форма кривой $I(V)$ (рис. \ref{pict I(V)}). Все это определяет выбор пригодных для опыта катодов.

    \section{Экспериментальная установка}
        \begin{figure}[!ht]
            \centering
            \includegraphics[width = 0.6\textwidth]{img/setup.png}
            \caption{Принципиальная схема экспериментальной установки}
            \label{exp_scheme}
        \end{figure}

        Свет от источника $S$ (обычная электрическая лампа накаливания) с помощью конденсора фокусируется на входную щель призменного монохроматора УМ-2, выделяющего узкий спектральный интервал, и попадает на катод фотоэлемента ФЭ.

        Фотоэлемент конструктивно представляет собой откачанный до высокого вакуума стеклянный баллон диаметром 25 мм и высотой 30 мм. Внутри баллона расположены два электрода: фотокатод и анод. Фотокатод представляет собой тонкую пленку металла, легированного элементами $Na$, $K$, $Sb$ и $Cs$ и расположенного на массивной металлической пластине. Анод фотоэлемента выполнен в виде пояска тонкой пленки, осажденной на внутренней части боковой поверхности вверху баллона. Такое расположение фотокатода и анода обеспечивает наиболее полный сбор на аноде электронов, эмитированных фотокатодом. Такой фотоэлемент обладает спектральной чувствительностью в области длин волн от 300 до 850 нм.

        Фототок, протекающий в фотоэлементе, мал, особенно при потенциалах $V$, близких к $V_0$ , и не может быть измерен непосредственно. Для его измерения используется усилитель постоянного тока.  Абсолютные значения фототока нам не нужны, поэтому он измеряется в относительных единицах цифровым вольтметром $V_2$ , подключенным к выходу усилителя. Эти показания пропорциональны величине измеряемого тока. Измерение тормозящего потенциала осуществляется с помощью цифрового вольтметра $V_1$.

        Контактная разность потенциалов между катодом и анодом мешает точному определению величины $V_0$, но не оказывает влияния на определение постоянной Планка, которая выражается через производную $dV_0 / d\omega$.

    \section{Ход работы}


\end{document}
