\documentclass[a4paper, 12pt]{article}
\usepackage[a4paper,top=1.5cm, bottom=1.5cm, left=1cm, right=1cm]{geometry}
\usepackage{cmap}					% поиск в PDF
\usepackage{mathtext} 				% русские буквы в формулах
\usepackage[T2A]{fontenc}			% кодировка
\usepackage[utf8]{inputenc}			% кодировка исходного текста
\usepackage[english,russian]{babel}	% локализация и переносы

\usepackage{amsmath,amssymb}
\usepackage{indentfirst}
\usepackage{longtable}
\usepackage{graphicx}
\usepackage{array}
\usepackage{float}

\usepackage{floatflt}
\usepackage{wrapfig}
\usepackage{siunitx} % Required for alignment
\usepackage{subfig}
\usepackage{multirow}
\usepackage{rotating}
\usepackage{caption}

\graphicspath{{.}}


\title{\begin{center}Лабораторная работа №5.2.2(2.3)\end{center}
Изучение спектров атомарного водорода и йода}
\author{Рожков А. В.}
\date{\today}

\begin{document}
    \pagenumbering{gobble}
    \maketitle
    \newpage
    \pagenumbering{arabic}

    \textbf{Цель работы:} исследовать спектральные закономерности в оптических спектрах водорода; по результатам измерений вычислить постоянные Ридберга для водорода; исследовать спектр поглощения паров йода в видимой области; по результатам измерения вычислить энергию колебательного кванта молекулы йода и энергию её диссоциации в основном и возбужденном состояниях.

    \textbf{В работе используются:} стеклянно-призменный монохроматор УМ-2, неоновая лампа, ртутная лампа ПРК-4, водородная лампа, кювета с кристаллами йода с лампой накаливания.

    \section{Теоретические сведения}
        \subsection*{Изучение спектра водорода}
            Атом водорода является простейшей квантовой системой, для которой уравнение Шрёдингера может быть решено точно. Это также верно для водородноподобных атомов, то есть атомов с одним электроном на внешней оболочке.
            Если считать ядро неподвижным, то эти энергетические уровни определяются выражением
            \begin{equation}
                E_n = - \frac{m_e (Z e^2)^2}{2\hbar^2}\frac{1}{n^2},
            \end{equation}
            где $n$ есть номер энергетического уровня, $Z$ есть зарядовое число ядра рассматриваемого атома, которое в случае атома водорода равно 1.\\
            При переходе электрона с $m$-го на $n$-й уровень излучается фотон с энергией
            \begin{equation}
                E_\gamma = E_m - E_n = \frac{m_ee^2}{2\hbar^2}Z^2\left(\frac{1}{n^2} - \frac{1}{m^2}\right) = Ry Z^2 (\frac{1}{n^2} - \frac{1}{m^2}).
            \end{equation}
            Длина волны  соответствующего излучения $\lambda_{mn}$ связана с номерами уровней следующим соотношением:
            \begin{equation}
                \label{eq:Ry}
                \frac{1}{\lambda_{mn}} =\frac{m_ee^2}{4\pi\hbar^3c}Z^2\left(\frac{1}{n^2}-\frac{1}{m^2}\right) = \text{R} Z^2 \left(\frac{1}{n^2}-\frac{1}{m^2}\right),
            \end{equation}
            где $\text{R} = \cfrac{m_ee^2}{4\pi\hbar^3c}$ есть постоянная Ридберга.

            В данной работе будет изучаться серия Бальмера атома водорода, в которой электроны совершают переходы с некоторого уровня $m$ на уровень $n = 2$.
        \subsection*{Спектр поглощения йода}
            В первом приближении энергия молекулы может быть представлена в виде:
            \begin{equation}
                E = E_{\text{эл}}+E_{\text{колеб}}+E_{\text{вращ}},
            \end{equation}
            где $E_{\text{эл}}$ есть энергия электронных уровней, $E_{\text{колеб}}$ есть энергия колебательных уровней, $E_{\text{вращ}}$ есть энергия вращательных уровней.

            В настоящей работе рассматриваются оптические переходы, то есть переходы, связанные с излучением фотонов в видимом диапазоне длин волн. Они соответствуют переходам между различными электронными состояниями. При этом также происходят изменения вращательного и колебательного состояний, однако в реальности ввиду малости характерных энергий вращательные переходы ненаблюдаемы.

            Более конкретно, изучаются переходы из колебательного состояния с номером $n_1$ основного электронного уровня с энергией $E_1$ в колебательное состояние с номером $n_2$ на электронный уровень с энергией $E_2$. Энергия таких переходов описывается формулой:
            \begin{equation}
                h \nu_{n_1,n_2}=(E_2-E_1)+h\nu_2(n_2+\dfrac{1}{2})-h \nu_1(n_1+\dfrac{1}{2}),
            \end{equation}
            где $h\nu_1$ и $h\nu_2$ суть энергии колебательных квантов на электронных уровнях с энергиями $E_1$ и $E_2$.

            При достаточно больших квантовых числах $n_1$ и $n_2$ колебательные уровни переходят в непрерывный спектр, что соответствует диссоциации молекулы. Наименьшая энергия, которую нужно сообщить молекуле в нижайшем колебательном состоянии, чтобы она диссоциировала, называется энергией диссоциации.

            В данной работе определяются энергии диссоциации на первых двух электронных уровнях.

            \begin{figure}[h!]
                \begin{center}
                \includegraphics[width = 0.5\textwidth]{img/potential_curves.png}
                \caption{Потенциальные кривые и характерные электронно-колебательные переходы}
                \label{potent_curves}
                \end{center}
            \end{figure}

            \subsubsection*{Серии Деландра в спектре йода}

                В данной работе изучается электронно-колебательный спектр поглощения паров йода I$_2$ в видимой области при температуре $T \approx 300\ \text{K}$. Основной вклад дают переходы  между колебательными подуровнями двух соседних электронных состояний:
                \[
                (1, n_1) \rightarrow (2, n_2),
                \]
                где индекс «1» обозначает основное электронное состояние, «2» — возбуждённое.

                Все возможные линии поглощения удобно разбить на \textit{серии Деландра}, каждая из которых соответствует фиксированному начальному колебательному уровню (например, $n_1 = 0$ или $n_1 = 1$), а конечный уровень $n_2 = 0, 1, 2, \ldots$ меняется:

                \begin{itemize}
                    \item 0-я серия: переходы из $n_1 = 0$ в $n_2 = 0, 1, 2, \ldots$;
                    \item 1-я серия: переходы из $n_1 = 1$ в $n_2 = 0, 1, 2, \ldots$.
                \end{itemize}

                При температуре около комнатной относительные заселённости колебательных уровней подчиняются распределению Больцмана:
                \[
                N_n \propto e^{-E_n/kT}.
                \]

                Расчёт показывает, что при $T \approx 300\ \text{K}$ выполняется примерно
                \[
                N_0 : N_1 : N_2 \approx 1 : \frac{1}{3} : \frac{1}{10},
                \]
                поэтому наибольший вклад дают 0-я и 1-я серии Деландра.

            \begin{figure}[h!]
                \begin{center}
                \includegraphics[width = 0.5\textwidth]{img/delandr.png}
                \caption{Спектр поглощения паров йода}
                \label{delandr}
                \end{center}
            \end{figure}

    \newpage

    \section{Экспериментальная установка}
        \subsection{Измерение серии Бальмера}

            \begin{figure}[h!]
                \begin{center}
                \includegraphics[width = 0.5\textwidth]{img/setup_1.png}
                \caption{Устройство монохроматора УМ-2}
                \label{setup1}
                \end{center}
            \end{figure}

            Для измерения длин волн спектральных линий используется монохроматор УМ-2, предназначенный для спектральных исследований в диапазоне от $0,38$ до $1,00$ мкм. Его устройство приведено на рис. \ref{setup1}.

            В качестве источника используется водородная лампа в виде H-образной трубки, питаемая от катушки Румкорфа

        \subsection{Измерение изотопического сдвига}

            \begin{figure}[h!]
                \begin{center}
                \includegraphics[width = 0.5\textwidth]{img/setup_1_2.png}
                \caption{}
                \label{setup2}
                \end{center}
            \end{figure}

            Для наблюдения небольшого изотопического сдвига между линиями водорода и дейтерия используется отражательная дифракционная решётка с высокой дисперсией и гониометр

        \subsection{Измерение спектр поглощения паров йода}

            \begin{figure}[h!]
                \begin{center}
                \includegraphics[width = 0.5\textwidth]{img/setup_2.png}
                \caption{Схема экспериментальной установки для получения спектра поглощения}
                \label{setup2}
                \end{center}
            \end{figure}

            Для получения спектра поглощения необходимы: 1) источник сплошного спектра - лампа накаливания;
            2) поглощающая среда - кювета с исследуемым веществом;
            3) спектральный прибор, регистрирующий спектр поглощения - монохроматор УМ-2.

\end{document}

