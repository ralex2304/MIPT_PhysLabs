\documentclass[a4paper, 12pt]{article}
\usepackage[a4paper,top=1.5cm, bottom=1.5cm, left=1cm, right=1cm]{geometry}
\usepackage{cmap}					% поиск в PDF
\usepackage{mathtext} 				% русские буквы в формулах
\usepackage[T2A]{fontenc}			% кодировка
\usepackage[utf8]{inputenc}			% кодировка исходного текста
\usepackage[english,russian]{babel}	% локализация и переносы

\usepackage{amsmath,amssymb}
\usepackage{indentfirst}
\usepackage{longtable}
\usepackage{graphicx}
\usepackage{array}
\usepackage{float}

\usepackage{floatflt}
\usepackage{wrapfig}
\usepackage{siunitx} % Required for alignment
\usepackage{subfig}
\usepackage{multirow}
\usepackage{rotating}
\usepackage{caption}

\graphicspath{{.}}

\title{\begin{center}Лабораторная работа №5.8.1\end{center}
Определения постоянных Стефана-Больцмана и Планка из анализа теплового излучения накаленного тела}
\author{Рожков А. В.}
\date{\today}

\begin{document}
    \pagenumbering{gobble}
    \maketitle
    \newpage
    \pagenumbering{arabic}

    \textbf{Цель работы:}  при помощи модели абсолютно чёрного тела провести измерения температуры оптическим пирометром с исчезающей нитью и термопарой; определить постоянные Планка и Стефана-Больцмана.

    \textbf{В работе используются:} оптический пирометр, модель абсолютно черного тела (АЧТ), лампа накаливания (с вольфрамовой нитью), керамическая трубка с кольцами из разных материалов, неоновая лампа, блок питания, цифровые вольтметры и амперметр.

    \section{Теоретические сведения}

        Для измерения температуры разогретых тел, удалённых от наблюдателя, применяют методы оптической пирометрии, основанные на использовании зависимости испускательной способности исследуемого тела от температуры. Различают три температуры, функционально связанные с истинной термодинамической температурой и излучательной способностью тела: радиационную $T_{rad}$, цветовую $T_{col}$ и яркостную $T_{br}$.

        В работе измеряется яркостная температура -- температура абсолютно чёрного тела, при которой его спектральная испускательная способность равна спектральной испускательной способности исследуемого тела при той же длине волны. Измерение яркостной температуры раскалённого тела производится при помощи оптического пирометра с исчезающей нитью, основанного на визуальном сравнении яркости раскалённой нити с яркостью изображения исследуемого тела. Равенство видимых яркостей, наблюдаемых через монохроматический светофильтр ($\lambda = 650$ нм), фиксируется по исчезновению изображения нити на фоне раскаленного тела. Яркостный метод измерения температуры основан, в соответствии с формулой Планка, на зависимости испускательной способности абсолютно чёрного тела от температуры и длины волны.

        \[ r^{\text{ачт}}_{\lambda, T} = \frac{2 \pi c^2 h}{\lambda^5} \frac{1}{\exp[hc/(\lambda k_{Б} T)] - 1} \]

        Шкалу прибора, измеряющего ток через нить, предварительно градуируют по абсолютно черному телу, термодинамическую температуру которого измеряют с помощью термопары. Если тело, температуру которого определяют, излучает как абсолютно черное тело, то, мы тем самым можем с помощью пирометра найти его температуру. Если же тело излучает иначе, то определенное значение температуры является яркостной температурой. Яркостная температура тела всегда ниже его термодинамической температуры.

        Среди нечерных тел выделяются так называемые серые тела, для которых характер распределения излучения совершенно подобен спектру абсолютно черного тела, но излучение ослаблено по сравнению с ним в $\varepsilon_T$ раз для любой длины волны при данной температуре тела $T$.

        Если предположить, что тело излучает как серое тело, то выражения для энергии излучения можно записать как

        \begin{equation*}
            W = \varepsilon_T S \sigma T^4, ~~ \sigma = \frac{2 \pi^5 k_{Б}^4}{15 c^2 h^3} = 5,67 \cdot 10^{-12} \frac{\text{Вт}}{\text{см}^2 \cdot K^4}
        \end{equation*}

        с учётом того, что температура тела намного больше температуры окружающей среды, и мощностью внешних потерь можно пренебречь.

        Из этого выражения можно экспериментально получить значение постоянной Планка:

        \begin{equation*}
            h = \sqrt[3]{\frac{2\pi^5k_{Б}^4}{15c^2\sigma}}
        \end{equation*}

        Положение максимума спектрального распределения определяется законом смещения Вина:

        \begin{equation*}
            \lambda_{max}T = b = 2.897 * 10^{-3}~м\cdotК
        \end{equation*}

        \begin{figure}[ht!]
            \begin{center}
                \includegraphics[width = 0.5\textwidth]{img/wien_law.png}
                \caption{Распределение энергии в спектре излучения абсолютно чёрного тела при различных температурах}
                \label{img:wien_law}
            \end{center}
        \end{figure}

    \section{Экспериментальная установка}

        Экспериментальная установка состоит из оптического пирометра 9, модели абсолютно черного тела (АЧТ), трех исследуемых образцов \\ ($18, 19, 20$), блока питания (1) и цифровых вольтметров В7-22А и В7-38.

        Оптический пирометр представляет собой зрительную трубу, внутри которой имеется накаливаемая нить, расположенная в плоскости изображения исследуемого раскаленного тела, а также темно-красный светофильтр ($\lambda = 650$ нм). Через окуляр одновременно наблюдается изображение исследуемого тела и раскаленной нити.

        Модель АЧТ представляет собой керамическую трубку диаметром $3$ мм и длиной $50$ мм, закрытую с одного конца и окруженную для теплоизоляции внешним кожухом. Нагрев трубки осуществляется намотанной на ней нихромовой спиралью, питаемой от источника тока. Полость трубки и особенно ее дно излучают практически как абсолютно черное тело. Температура модели АЧТ измеряется хромельалюмелевой термопарой, один спай которой вмонтирован в дно трубки, а другой находится при комнатной температуре на клемме цифрового вольтметра В7-38, измеряющего ЭДС термопары.

        В работе исследуются три образца. Один образец выполнен в виде керамической трубки с набором колец из различных материалов, нагреваемой изнутри нихромовой спиралью. Другой исследуемый образец -- вольфрамовая нить электрической лампочки. Она питается от источника 1, когда переключатель 6 находится в положении 3. Сила тока через вольфрамовую нить измеряется с помощью прибора В7-22А (15). Падение напряжения на самой нити измеряется непосредственно вольтметром В7-22А (16). Таким образом, зная показания обоих приборов, можно определить мощность, потребляемую нитью лампочки.

        \begin{figure}[ht!]
            \begin{center}
                \includegraphics[width = 0.5\textwidth]{img/setup.png}
                \caption{Схема экспериментальной установки: 1 -- блок питания; 2 -- тумблер включения питания пирометра и образцов; 3 -- тумблер нагрева нити пирометра: <<Быстро>> -- вверх, <<Медленно>> -- вниз; 4 -- кнопка <<Нагрев нити>>; 5 -- кнопка <<охлаждение нити>>; 6 -- тумблер переключения образцов; 7 -- регулятор мощности нагрева образцов; 8 -- окуляр пирометра; 9 -- корпус пирометра; 10 -- объектив пирометра; 11 -- переключение диапазонов: 700-1200 C -- вниз, 1200-2000 C -- вверх; 12 -- ручка перемещения красного светофильтра; 13 -- регулировочный винт; 14 -- вольтметр (напряжение на лампе накаливания); 15 -- амперметр (ток через образцы); 16 -- вольтметр в цепи термопары; 17 -- модель АЧТ; 18 -- трубка с кольцами из материалов с разной излучательной способностью; 19 -- лампа накаливания; 20 -- неоновая лампочка}
                \label{setup1}
            \end{center}
        \end{figure}

        Для обработки данных необходима также зависимость между яркостной и термодинамической температурами вольфрама, приведённая на рис. \ref{img:wolfram_brightness}.

        \begin{figure}[ht!]
            \begin{center}
                \includegraphics[width = 0.5\textwidth]{img/wolfram_brightness.png}
                \caption{}
                \label{img:wolfram_brightness}
            \end{center}
        \end{figure}

    \newpage

    \section{Ход работы}

\end{document}

