\documentclass[a4paper, 12pt]{article}
\usepackage[a4paper,top=1.5cm, bottom=1.5cm, left=1cm, right=1cm]{geometry}
\usepackage{cmap}					% поиск в PDF
\usepackage{mathtext} 				% русские буквы в формулах
\usepackage[T2A]{fontenc}			% кодировка
\usepackage[utf8]{inputenc}			% кодировка исходного текста
\usepackage[english,russian]{babel}	% локализация и переносы

\usepackage{amsmath,amssymb}
\usepackage{indentfirst}
\usepackage{longtable}
\usepackage{graphicx}
\usepackage{array}
\usepackage{float}

\usepackage{floatflt}
\usepackage{wrapfig}
\usepackage{siunitx} % Required for alignment
\usepackage{subfig}
\usepackage{multirow}
\usepackage{rotating}
\usepackage{caption}

\graphicspath{{.}}


\title{\begin{center}Лабораторная работа №5.4.1\end{center}
Определение энергии $\alpha$-частиц по величине их пробега в воздухе}
\author{Рожков А. В.}
\date{\today}

\begin{document}
    \pagenumbering{gobble}
    \maketitle
    \newpage
    \pagenumbering{arabic}

    \textbf{Цель работы:} измерить пробег $\alpha$-частиц в воздухе тремя способами:
    с помощью торцевого счётчика Гейгера, сцинтилляционного счетчика и ионизационной камеры,
    - по полученным величинам определяется энергия частиц

    \textbf{В работе используются:} счётчик Гейгера, сцинтилляционный счетчик, ионизационная камера.

    \section{Теоретические сведения}

        При $\alpha$-распаде исходное родительское ядро испускает ядро гелия и превращается в дочернее ядро, число протонов и число нейтронов уменьшается на две единицы. Функциональная связь между энергией $\alpha$-частицы $E$ и периодом полураспада радиоактивного ядра $T_{1/2}$ хорошо описывается формулой
        \begin{equation*}
             \lg T_{1/2} = \frac{a}{\sqrt{E}} + b.
        \end{equation*}

        Экспоненциальный характер этого процесса возникает вследствие экспоненциального затухания волновой функции в области под барьером, где потенциальная энергия больше энергии частицы.


        Экспериментально энергию $\alpha$-частиц удобно определять по величине их пробега в веществе.
        Для описания связи между энергией $\alpha$-частицы и ее пробегом пользуются эмпирическими соотношениями. В диапазоне энергий $\alpha$-частиц от $4$ до $9$ МэВ эта связь хорошо описывается выражением
        \begin{equation}
            R = 0,32 E^{3/2},
        \end{equation}
        где пробег $\alpha$-частиц в воздухе $R$ (при 15 $^\circ C$ и атмосферном давлении) выражается в см, а энергия частицы $E$ в МэВ.

        \begin{figure}[h!]
            \centering
            \includegraphics[width = 0.5\linewidth]{img/dn_dx.png}
            \caption{Зависимость числа $\alpha$-частиц от глубины их проникновения в вещество}
            \label{}
        \end{figure}

        При малых глубинах число частиц не меняется с расстоянием. В конце пути это число не сразу обрывается до нуля, а приближается к нему постепенно. Как видно из кривой $dN/dx$, большая часть $\alpha$-частиц останавливается в узкой области, расположенной около некоторого значения $x$, которое называется средним пробегом $R_{\text{ср}}$. Иногда вместо $R_{\text{ср}}$ измеряются экстраполированное значение $R_{\text{э}}$.

        В силу размытия и смещения брэгговского пика из-за угловой расходимости пучков частиц, лучшей оценкой пробега оказывается экстраполированный пробег.

    \section{Экспериментальная установка}

        \subsection{Счётчик Гейгера}

            Для определения пробега $\alpha$-частиц с помощью счетчика радиоактивный источник помещается на дно стальной цилиндрической бомбы, в которой может перемещаться торцевой счетчик Гейгера. Его чувствительный объем отделен от наружной среды тонким слюдяным окошком, сквозь которое могут проходить $\alpha$-частицы.

            Импульсы, возникающие в счетчике, усиливаются и регистрируются пересчетной схемой. Путь частиц в воздухе зависит от расстояния между источником и счетчиком. Перемещение счетчика производится путем вращения гайки, находящейся на крышке бомбы. Расстояние между счетчиком и препаратом измеряется по шкале, нанесенной на держатель счетчика.

            \begin{figure}[h!]
                \centering
                \includegraphics[width = 0.3\linewidth]{img/geiger.png}
                \caption{Счётчик Гейгера}
                \label{}
            \end{figure}

        \subsection{Сцинтилляционный счётчик}

            Установка состоит из цилиндрической камеры, на дне которой находится исследуемый препарат. Камера герметично закрыта стеклянной пластинкой, на которую с внутренней стороны нанесен слой люминофора. С наружной стороны к стеклу прижат фотокатод фотоумножителя. Оптический контакт ФЭУ-стекло обеспечивается тонким слоем вазелинового масла.

            Сигналы с фотоумножителя через усилитель поступают на пересчетную установку. Расстояние между препаратом и люминофором составляет 9 см, так что $\alpha$-частицы не могут достигнуть люминофора при обычном давлении. Определение пробега сводится к измерению зависимости интенсивности счета от давления в камере.

            \begin{figure}[h!]
                \centering
                \includegraphics[width = 0.3\linewidth]{img/scint.png}
                \caption{Установка для измерения пробега $\alpha$-частиц с помощью сцинтилляционного счетчика}
                \label{}
            \end{figure}

        \subsection{Ионизационная камера}

            Ионизационная камера -- прибор для количественного измерения ионизации, произведенной заряженными частицами при прохождении через газ. Камера представляет собой наполненный газом сосуд с двумя электродами. Сферическая стенка прибора служит одним из электродов, второй электрод вводится в газ через изолирующую пробку. К электродам подводится постоянное напряжение от источника ЭДС. Заполняющий сосуд газ сам по себе не проводит электрический ток, возникает он только при прохождении быстрой заряженной частицы, которая рождает в газе на своем пути ионы. Поместим на торец внутреннего электрода источник ионизирующего излучения, заполним объем камеры воздухом.

            Прохождение тока через камеру регистрируется посредством измерения напряжения на включенном в цепь камеры сопротивлении $R$. При небольших давлениях газа $\alpha$-частицы передают часть энергии стенкам камеры. По достижении давления $P_0$ все они заканчивают свой пробег внутри газа, и дальнейшее возрастание тока прекращается. Для определения давления $P_0$ чаще всего пользуются методом экстраполяции, продолжая наклонный и горизонтальный участки кривой до пересечения. Найденный таким образом пробег затем должен быть приведен к нормальному давлению и температуре 15 $^oC$.

            \begin{figure}[h!]
                \centering
                \includegraphics[width = 0.4\linewidth]{img/ion.png}
                \caption{Установка для измерения пробега $\alpha$-частиц с помощью ионизационной камеры}
                \label{}
            \end{figure}


\end{document}

